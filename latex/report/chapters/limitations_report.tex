\chapter{Limitations}
\label{ch:limitations_report}

In this chapter, let us investigate the limitations with the wireframe tool used and the user study process. In this thesis, we have used Balsamiq tool to create designs for solutions ideas to research questions. Although Balsamiq is a widely used tool in designers work environment and these kinds of research scenario. However, it showed some limitations in specific to our context with few research questions. We see one while dealing with some solution ideas that need an animation effect. The current tool version did not provide the animation effects. So as a workaround, we have demonstrated the task to mimic the animation effect using multiple mock-up screens. \\ \\

The other limitation observed is with jumping of mock-up screens while clicking out of context as from tool perspective it denotes moving to next mock-up screen. However, from a user perspective, it surprises when user just out of habit clicks randomly out of a hyperlink or button that needs transition. \\ \\

Now, with the user study process, while the session is going on, there is a chance for user getting primed. The exposure to one stimulus influences the following behaviour of the user. In psychology, this is called Priming. \cite{priming} In our context, for example, it means teaching the user what we need them to know at the beginning of a task. Using the pre-test questionnaire, it helps to determine the user background knowledge and based on it; we give a short explanation on the topic, for better understanding of what this user study is all about. \\ \\

There is also chance of recency bias crept into the user study sessions. As we deal with humans, there is a high chance for cognitive bias. In particular, to our context, the users tend to think about the recent design and might outweigh their opinions on it while providing feedback. This bias is taken care of while changing the order of designs shown to the user using a technique called Latin Square partition. \cite{lsp} \\ \\

With given constraints, we have done a closed user study research. In which, users are asked to perform tasks and interviewed in a typical lab setting. But in reality, the users could happen to work in a more relaxed manner and their work environments. \\ \\

Other limitation is seen with recruiting the participants. We have met the minimum threshold of user study participants as per Usability Engineering experts, i.e., 5. \cite{five} In all three cycles conducted where first and third cycle has 5 participants and second cycle has 7 participants. This users recruitment has been a challenge with available sources. However, it would be more representative and shows soundness in results with qualitative data with higher the participants. Nevertheless, with given limitations in this thesis, we have focused more on qualitative feedback. So the user study process is more inclined towards formative study. \\ \\