\chapter{Introduction}
\label{ch:introduction_report}

Software is everywhere in all walks of life as seen with the development of ‘Internet of Things’ as an example. The effectiveness of software development relies on bug-free coding. In our day-to-day progress in coding leads to complexity of software, which brings a broader scope to introduce bugs and vulnerabilities. The presence of bugs impacted a significant loss to the extent of \$1.1 Trillion in 2016. \cite{report} There are several static analysis tools available in the market to address these primary issues. However, in the latest surveys by Maria \etal{} \cite{CB16} and Johnson \etal{} \cite{JSMB13}, it is noticed that software developers are not happy with the effectiveness and usability of static analysis tools.
\\ \\

In general, a software development organisation used to use a single tool in the beginning in their SDLC (Software Development Life Cycle) process. Later on, when different static analysis tools came into the market has a reputation for different capabilities on findings of bugs, as an example are emerged. Therefore, organisations considered adding multiple tools into their development cycle. The other reason could also be some tools are free and open-source, which made management team to add for more significant advantage. The advantages could be such as reducing false positives by recognising a bug reported by different tools, maximising the possibility of detection of bugs. These advantages lead to a scenario of using multiple static analysis tools for a single software project.
\\ \\

In the scenario where an organisation uses different tools, it leads to a disruptive workflow of the development process. This kind of workflow brings a new challenge on how to make theses tools integrate to the existing SDLC in a less disruptive way by improving the respective user interface in terms of usability. This scenario opens a new opportunity/challenge, which requires research, and this thesis aims to address it.
\\ \\

\section{Problem Statement}

\subsection{How to Integrate the Results of Multiple Static Analysis Tools in a Unified User Interface?}

The overall main aim of the thesis is about, “ \textit{How to integrate the results of multiple static analysis tools in a unified user interface?} ”. We broke this question down into different research questions during the literature review. We selected the three essential research questions based on the scope and time limits of the thesis work. \\ \\

\noindent\textbf{Research Question 1}: \\ How to display results of the same code base from different analysis tools? \\
\textbf{Research Question 2}: What feedback works to know that bug fixing is on-going? \\
\textbf{Research Question 3}: How to carry traceability of bug fixing? \\ \\

We explain the research questions at \autoref{ch:relatedwork_report}. To answer each research question, we design the user interface with novel ideas and also by researching the different software engineering disciplines tackling a similar issue. We evaluate the developed prototype with the ideas brainstormed during research with software developers. As part of the evaluation, we assess the usability \cite{usability} of the user interface which helps to notice new usability problems and then address them in the next following iteration of the ‘User Experience Design cycle’ \cite{UXD} which is the essence of ‘Human-Centred Design’ \cite{hcd}. The problems gathered in an evaluation are considered as requirements for a new design and the process repeats. This methodology leads to multiple iterations of the ‘User Experience Design cycle’. It follows this approach for all three research questions. The primary contribution of the thesis is to make sure the ideas tested are valid. \\ \\



\section{Outline}

This Introduction chapter mentioned what the thesis is all about. We structure the remaining part of the report, as mentioned below. \\ \\

\noindent Chapter \ref{ch:background_report} explains the key concepts such as ‘Static Analysis’, ‘Usability’ and ‘Wireframe’ which are necessary to understand the work of this thesis. \\
Chapter \ref{ch:literaturereview_report} mentions the essential aspects from other software engineering disciplines studied as part of the thesis work. \\
Chapter \ref{ch:motivation_report} discusses the current research findings and the need for doing the thesis. \\
Chapter \ref{ch:relatedwork_report} explains what current research is up to in using multiple analysis tools and what current tools do related to our primary research questions. \\
Chapter \ref{ch:researchmethodology_report} mentions about the UX Design research methodology and what are the critical concepts considered during evaluation. \\
Chapter \ref{ch:objectives_report} shows the original working plan of the thesis and what we need to follow. \\
Chapter \ref{ch:approaches_report} explains how we approached the research questions and their evaluation with solution ideas. \\
Chapter \ref{ch:ux1-cycle_report} explains what the sub research questions considered in the first iteration are, how they are evaluated with user study protocol and explains the potential solution ideas. \\
Chapter \ref{ch:ux2-cycle_report} explains what the sub research questions considered in the second iteration are, how they are evaluated with user study protocol and explains the potential solution ideas. \\
Chapter \ref{ch:ux3-cycle_report} explains what the sub research questions considered in the third iteration are, how they are evaluated with user study protocol and explains the potential solution ideas. \\
Chapter \ref{ch:limitations_report} mentions the overall limitations with the thesis work. \\
Chapter \ref{ch:futurework_report} shows what we could do as the next step in research covered so far within this thesis scope. \\
Chapter \ref{ch:conclusion_report} mentions the final thoughts of thesis work. \\ \\

\let\cleardoublepage\clearpage
