\chapter{Future Work}
\label{ch:futurework_report}

The future work ideas discussed here. \\ \\

With given scope and time limits of present thesis work, we have picked up important among the research questions formed while brainstorming on the thesis topic theme. So, here are some list of research questions that we consider for future work. \\ \\

In the scope of the first primary research question, i.e., display of results, here are few more sub research questions.
\begin{enumerate}
\item Would having tabs help scale the tools visibility with bugs results?
\item Do graphs help in understanding the bugs reported?
\item Will the user need to have graphs for separate tools or one graph combining the tools selected?
\item In scenarios of having unique (single) graph for multiple tools, does user expect to have it shown all the tools integrated by default or need of filters for tools selection? \\ \\
\end{enumerate} 

In the scope of the third primary research question, i.e., traceability of bug fixing here is one more sub research question. \\ \\
\noindent Does having graphs (example: histograms) at particular part of code with commits related bug fixes help the user to trace? \\ \\

We also consider to add new main research question looking into aspect of teamwork in fixing bugs. \\ \\
\noindent RQ 4 - How is teamwork facilitated in bug fixing in context of multiple tools? \\ \\

Also, it would be interesting to evaluate the complete prototype by covering the sub research questions into a single prototype. As we have promising finalised solution ideas through evaluation for each different scenarios in using multiple tools with a single interface, we look forward to develop a real interface when feasible technology stack and resources available. \\ \\

As mentioned in Chapter \ref{ch:limitations_report} about the design tool, it would be good to consider if there is an alternative tool that could help with animation features. It makes it more easier to understand than the present approach.