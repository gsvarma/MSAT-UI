\chapter{Conclusion}
\label{ch:conclusion}


As software has become a part of a day to day life, bug-free software is quite necessary. For which, static code analysis tools are used to identify and mitigate bugs during software development life cycle. Although these are essential tools, in terms of usability they lack the attention of developers. Further, the use of multiple static code analysis tools is increasing for obvious reasons like to ensure the bug priority, increase coverage area etc. On the other hand, there is a standardisation process going on with SARIF format and this promises to have generated results in the same format from different analysis tools. \\ \\

Thereby, this shows the future scope of ongoing research with multiple tools usage. So, when the integration of results from different tools happens, it is necessary to have single interface. This thesis aims to address this issue with three different research questions i.e., \\ 1. How to display results of the same codebase from different analysis tools? \\
2. What feedback works to know that the bug fixing is on-going? \\
3. How to carry traceability of bug fixing? \\ \\

The research questions are answered by studying different software engineering disciplines and adapt the possible techniques if any. Also, the developers' feedback is analysed through user experience design cycle to make sure the designed prototypes are usable enough to overcome the issues. Hence, the thesis work ensures the applicability of results examined.