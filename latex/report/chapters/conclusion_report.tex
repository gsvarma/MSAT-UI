\chapter{Conclusion}
\label{ch:conclusion_report}

As software has become a part of day-to-day life, bug-free software is necessary. For which, static code analysis tools are used to identify and mitigate bugs during the software development life cycle. Although these are essential tools, in terms of usability, they lack the attention of developers. Further, the use of multiple static code analysis tools is increasing for obvious reasons such as to ensure the bug priority, increase coverage area. On the other hand, there is a standardisation process going on with a SARIF format, \cite{sarif-git} and this promises to have generated results in the same format from different analysis tools. \\ \\

These reasons show the future scope of on-going research with multiple tools usage. So, when the integration of results from different tools happens, it is necessary to have a single interface. This thesis addressed this issue with three different primary research questions, i.e., \\ 
\begin{enumerate}
\item  How to display results of the same code base from different analysis tools?
\item  What feedback works to know that bug fixing is on-going?
\item  How to carry traceability of bug fixing? \\ \\
\end{enumerate}
The research questions are answered by studying different software engineering disciplines and adapting the possible techniques if any. Also, the developers’ feedback is analysed through the user experience design cycle to make sure the designed prototypes are usable enough to overcome the issues. Hence, the thesis work ensured the applicability of the results examined. \\ \\

We find the examined solution ideas to be good enough to consider as a stepping stone in the development of multiple static analysis tool user interfaces. \\ \\

\let\cleardoublepage\clearpage