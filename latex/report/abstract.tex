Software is everywhere in every walk of our life with new advancements such as 'Internet of Things'. The reports show there has been a trillion of dollars loss recently because of the presence of bugs in software.  Static analysis plays a major role in software development to find bugs and any vulnerabilities in code. There are different static analysis tools available in the market. However, it is found out in different surveys about why the static analysis tools are not as efficient as expected by software developers. The problems found out to be bad warning messages, complex user interface etc. Although there are solutions in industries to workaround while using a single tool but results in new issues when using multiple tools. In recent research, it is found out that in a typical software development organisation, they use multiple tools including legacy tools used in nightly builds as an example. On the other hand with ongoing research trends in using multiple static analysis tools namely Tricorder, Parfait etc. shows the opportunity and importance of developing a single interface for multiple tools. This thesis aims to address the scenario where a developer works with different tools and how adaptive could be the user interface. The novel ideas including approaches adapted from different software engineering disciplines are evaluated through user experience design cycle. Designs are made with assimilated ideas and prototypes are build using a wireframe tool. The usability aspect of the proposed ideas is considered during the evaluation phase. The target users for this evaluation are experienced software developers which ensures the applicability of this thesis work. \\ \\
\textbf{\textit{Keywords}}: Static Analysis, Usability, Wireframe, User Experience Design
