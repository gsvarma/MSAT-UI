Software is omnipresent with new advancements such as ‘Internet of Things’. Recently, the reports show that there has been a trillion-dollar loss because of the presence of bugs in software. Static analysis plays a significant role in software development to find bugs and any vulnerabilities in code. There are different static analysis tools available in the market. However, various surveys found out why the static analysis tools are not as efficient as expected by software developers. The problems found out to be such as bad warning messages, complex user interface. Although there are solutions in industries to workaround while using a single tool, it results in new issues when using multiple tools. Recent research, it is found that in a typical software development organisation, they use multiple tools, including legacy tools used in nightly builds as an example. On the other hand, with on-going research trends in using multiple static analysis tools, such as Tricorder, Parfait shows the opportunity and importance of developing a single interface for multiple tools. This thesis aims to address the scenario where a developer works with different tools and how adaptive it could be the user interface. The novel ideas, including approaches adapted from different software engineering disciplines, are evaluated through the user experience design cycle. Designs made with assimilated ideas, and prototypes are built using a wireframe tool. With scientific research on the various scenarios categorised into three main research questions, i.e., displaying the results, feedback for bug-fixing, and traceability, results an interesting solution ideas. Single list with table format having a column with tool icons found to suffice in displaying the results in the analysis view and a universal standard bug icon instead of showing individual tool icons in code view. The feedback ideas such as animated icons, progress bar and pending status popup are found to be useful in combination. Alerts feedback is also felt promising, especially during bug fix failures. To trace the changes in a section of code during a bug fix, a single-window with table view helped users to perceive the results. Evaluation phase considers the usability aspect of the proposed ideas. The target users for this evaluation are software developers who has good experience in programming, which ensures the applicability of this thesis work. Thereby, the evaluated solution ideas acts as a forward step to the development of the interface. \\ \\
\textbf{\textit{Keywords}}: Static Analysis, Usability, Wireframe, User Experience Design
