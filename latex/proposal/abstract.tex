The Static Analysis plays a major role in Software Development to find bugs and any vulnerabilities in code. There are different Analysis tools available in market. However, it is found out in different surveys about why the tools are not efficient as expected by Software Developers. With the advent of new Static Analysis approaches called as Incremental Analysis, Layered Analysis or Just-in-Time Analysis which aims to compute the analysis in short time. This challenges even further to the developer as to how to integrate such usage of tools in real time scenario. Also, in a typical Software Development Organisations they use multiple tools including legacy tools used in Nightly builds as an example. This Thesis aims to address that scenario where developer works with different tools and how responsive could be the User Interface. The novel ideas including approaches adapted from different Software Engineering disciplines are evaluated through Usability Design cycle. Aspects such as Soundness, Completeness and Scalability are considered during evaluation phase. The target users for this evaluation are experienced Software Developers which ensures the applicability of this Thesis work. \\
\textbf{\textit{Keywords}}: Static Analysis, Incremental Analysis, Layered Analysis, Responsiveness, Wireframe, User Experience Design
