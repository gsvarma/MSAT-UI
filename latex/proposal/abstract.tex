Static analysis plays a major role in software development to find bugs and any vulnerabilities in code. There are different static analysis tools available in the market. However, it is found out in different surveys about why the static analysis tools are not as efficient as expected by software developers. The problems found out to be  In recent research, it is found out that in a typical software development organisation, they use multiple tools including legacy tools used in nightly builds as an example. On the other hand with ongoing research trends in using multiple static analysis tools and standardisation of results format i.e., SARIF shows the opportunity and importance of developing a single interface for multiple tools. This Thesis aims to address the scenario where a developer works with different tools and how adaptive could be the user interface. The novel ideas including approaches adapted from different software engineering disciplines are evaluated through user experience design cycle. The usability aspect of the proposed ideas is considered during the evaluation phase. The target users for this evaluation are experienced software developers which ensures the applicability of this Thesis work. \\ \\
\textbf{\textit{Keywords}}: Static Analysis, Usability, Wireframe, User Experience Design
