\chapter{Introduction}
\label{ch:introduction}

The effectiveness of Software Development relies on bug free coding. In our day to day progress in coding leads to complexity of software which brings a broader scope for bugs and vulnerabilities that could be introduced easily. There are many Static Analysis tools available in market to address these primary issues. However in latest surveys by Maria \etal{} \cite{CB16} and by Johnson \etal{} \cite{JSMB13} it is noticed that Software Developers are not quite happy with effectiveness and usability of Static Analysis tools. This brings the scope for improvement of Static Analysis tools and the paper by Nguyen Quang Do \etal{} \cite{NB18} introduces how Gamifying the bug fixing process could enhance the usability of Static Analysis tool. Making User Interface Responsive is presented as one of the challenge in the paper. This Thesis work aims to address that challenge as an overview. 
\\ \\

In general, a Software Development Organisation used to use a single tool in the beginning in their Software Development Life Cycle. Later on, when different Static Analysis tools came into market having reputation for different capabilities on findings of bugs, as an example are emerged then Organisations considered to add multiple tools into their Development cycle. The other reason could also be some tools are free and open source which made Management team to simply add for greater advantage. The advantages could be reducing false positives by recognising a bug reported by different tools, maximise the possibility of detection of bugs etc. This lead to scenario of using multiple static analysis tools for a single software project.
\\ \\

The Static Analysis tools are evolving from time to time in a direction of reducing computation time. To name a few, they are called Incremental Analysis, Layered Analysis and Just-in-Time Analysis etc. For example, a tool named Cheetah which is introduced as a Just-in-Time Taint Analysis for Android Applications scans a project in less than a second. An other example, a tool with incremental Static Analysis using Path Abstraction named iSATURN proved in the evaluation phase as improvement in scan time by 32 percent over previous SATURN algorithm. 
\\ \\

In scenario where an Organisation decides to use different tools as such and especially a legacy tool in combination with newly evolved tool, it leads to disruptive workflow of development process. This brings new challenge on how to make theses tools integrate to the existing Software Development Life Cycle in less disruptive way by improving the respective User Interface in terms of responsiveness. This opens a new opportunity / challenge which requires Research and thereby this Thesis aims to address it.
\\ \\

\section{Problem Statement}

The overall main aim of the Thesis is about, " \textit{How to integrate the results of multiple Static Analysis tools?} ". This question was broken down into different Research Questions during Literature Review / Preparatory phase. The 3 important Research Questions are selected with respect to scope and time limits of the Thesis work. \\ \\

\noindent\textbf{Research Question 1}: How to display results of the same codebase from different analysis tools? \\
\textbf{Research Question 2}: What feedback works to know that the bug fixing is on-going? \\
\textbf{Research Question 3}: How to carry traceability of bug fixing? \\ \\

The Research Questions are explained in detail at Motivation chapter \ref{ch:motivation}. These are evaluated by developing prototypes with the ideas brainstormed during Research and assessed with Software Developers, the responsiveness of User Interface. So, as part of primary contribution of the Thesis is to make sure the ideas evaluated are valid.

\section{Outline}

This Introduction chapter mentioned the What aspect of doing the Thesis. The remaining part of the Proposal is structured as follows: \\ \\

\noindent Chapter \ref{ch:background} explains key concepts necessary to understand the work of this Thesis. \\ 
Chapter \ref{ch:motivation} mentions the Why aspect of doing the Thesis. \\
Chapter \ref{ch:objectives} overviews the goals of the Thesis work. \\
Chapter \ref{ch:approaches} mentions the How aspect of the Thesis which explains the way the challenges addressed or attempts to answer the Research Questions. \\
Chapter \ref{ch:evaluationplan} evaluates the solution ideas for the challenges or answers to research to Research Questions. \\
Chapter \ref{ch:timeplan} shows the calendrical time plan of doing the Thesis work. \\ \\

and finally Bibliography with cited references during the Thesis.

\let\cleardoublepage\clearpage
