\chapter{Introduction}
\label{ch:introduction}

Software is everywhere in all walks of life as you see with development of Internet of Things as an example. The effectiveness of Software Development relies on bug free coding. In our day to day progress in coding leads to complexity of software which brings a broader scope for bugs and vulnerabilities that could be introduced easily. The presence of bugs impacted a major loss to an extent of  \$1.1 Trillion in 2016. \cite{report} There are many static analysis tools available in market to address these primary issues. However in latest surveys by Maria \etal{} \cite{CB16} and by Johnson \etal{} \cite{JSMB13} it is noticed that software developers are not quite happy with effectiveness and usability of static analysis tools.
\\ \\

In general, a software development organisation used to use a single tool in the beginning in their SDLC (Software Development Life Cycle) process. Later on, when different static analysis tools came into market having reputation for different capabilities on findings of bugs, as an example are emerged then organisations considered to add multiple tools into their development cycle. The other reason could also be some tools are free and open source which made management team to simply add for greater advantage. The advantages could be reducing false positives by recognising a bug reported by different tools, maximise the possibility of detection of bugs etc. This lead to scenario of using multiple static analysis tools for a single software project.
\\ \\

In scenario where an organisation decides to use different tools it leads to disruptive workflow of development process. This brings new challenge on how to make theses tools integrate to the existing SDLC in less disruptive way by improving the respective user interface in terms of usability. This opens a new opportunity / challenge which requires Research and thereby this Thesis aims to address it.
\\ \\

\section{Problem Statement}

\subsection{How to Integrate the Results of Multiple Static Analysis Tools in a Unified User Interface?}

The overall main aim of the Thesis is about, " \textit{How to integrate the results of multiple static analysis tools in a unified user interface?} ". This question was broken down into different Research Questions during Literature Review / Preparatory phase. The three important Research Questions are selected with respect to scope and time limits of the Thesis work. \\ \\

\noindent\textbf{Research Question 1}: \\ How to display results of the same codebase from different analysis tools? \\
\textbf{Research Question 2}: What feedback works to know that the bug fixing is on-going? \\
\textbf{Research Question 3}: How to carry traceability of bug fixing? \\ \\

The research questions are explained in detail at chapter \ref{ch:motivation}. To answer each research question, the user interface is designed with novel ideas and also by researching into  the different software engineering disciplines tackling a similar issue. The developed prototype with the ideas brainstormed during research is evaluated with software developers. As part of the evaluation, the usability  \cite{usability} of the user interface is assessed and therefore new usability problems could be noticed which requires to be addressed in the next following iteration of the User Experience Design cycle \cite{UXD} which is the essence of Human Centered Design \cite{hcd}. The problems gathered in an evaluation are considered as requirements for new design and the process repeats. This leads to multiple iterations of the User Experience Design cycle. This approach is followed for all the three research questions. The primary contribution of the Thesis is to make sure the ideas evaluated are valid. \\ \\

\section{Outline}

This Introduction chapter mentioned what the Thesis is all about. The remaining part of the Proposal is structured as follows: \\ \\

\noindent Chapter \ref{ch:background} explains the key concepts such as 'Static Analysis', 'User Experience Design' and 'Wireframe' which are necessary to understand the work of this Thesis. \\ 
Chapter \ref{ch:motivation} discusses what the current research findings and the need of doing the Thesis. \\
Chapter \ref{ch:objectives} overviews the goals of the Thesis work. \\
Chapter \ref{ch:approaches} explains the way the challenges addressed or attempts to answer the Research Questions using User Experience Design cycle. \\
Chapter \ref{ch:evaluationplan} shows how the user interface designs are evaluated with the solution ideas for the challenges or answers to Research Questions. \\
Chapter \ref{ch:timeplan} shows the time plan of accomplishing the Thesis work. \\ \\


\let\cleardoublepage\clearpage
