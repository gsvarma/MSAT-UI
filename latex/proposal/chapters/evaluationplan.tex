\chapter{Evaluation Plan}
\label{ch:evaluationplan} 

This chapter explains how the evaluation of the solution ideas to Research Questions are done. \\ \\

The primary thing to be done is User Study as per the protocol mentioned below and then assimilate the results gathered during User Study into well-formed answered to our research questions. \\ \\

As the order of prototypes with different solution ideas presented in evaluation could influence the user as they tend to learn. Therefore, the order is changed for different segments of users which could lead to a qualitative result. \\ \\

Here is the draft for User Study protocol. \\

The User Study is performed following this protocol as guidelines. \\ \\

\textbf{Phase 1: Set up study environment} \\

The solution ideas for the Research Questions are assimilated into best 3 prototypes prepared using Mockup tool i.e., Balsamiq. These prototypes as exported as pdf and saved at Online drive which can be given access to the user once he agrees to participate in the study. \\\\

\textbf{Phase 2: Organize appointments with users} \\

An invitation will be sent to users typically software developers and preferably who use Static Analysis tools or at least having minimum experience. This ensures the results to be qualitative.
A Consent form is signed to make sure the collected user data can be used for this user study safely without violating user privacy. \\\\

\textbf{Phase 3: Introduction} \\

The User Study is introduced once the user shows up at the appointment. Once again, what this user study about and what data is collected during the process. \\\\

\textbf{Phase 4: General Questions} \\

Video recording is started. The user is asked with questions related to his software development experience and with Static Analysis tools.  \\\\\\

\textbf{Phase 5: Performance of tasks / Observing the usage} \\

The user is given with the task to fix a bug in a way that one could do on an interface. The task would be the following as an example; \\\\
-	Select the project \\
-	Observe the bugs notification \\
-	Select a bug \\
-	After observing the code editor, \\
click on an option on screen to indicate the user has fixed the bug \\

During this process, the user gives feedback on how he/she feels by the responsive nature of the Analysis tool. This could be analysed with predetermined questions to make results quantitative. \\

This process is repeated for the three prototypes.\\\\

\textbf{Phase 6: Wrap up} \\

The user is thanked for participation and recorded video is saved for data extraction.\\\\

\textbf{Phase 7: Data Extraction} \\

The questionnaire asked during the user study session is filled out in form for analysis in comparison to other users’ responses.\\

This helps to conclude whether the researched Static Analysis Tools User Interfaces are responsive as per the user expectations and usage.\\

\let\cleardoublepage\clearpage