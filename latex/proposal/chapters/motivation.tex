\chapter{Motivation}
\label{ch:motivation}

In general, the setup of most of the research done in the area of Static Code Analysis is like assuming a single project in an organisation. Further, they assume there is a single person working on a single project with a single tool tackling a single type of problems.  Somehow, the assumptions are made so singular to address a specific issue in their research. \\ \\
However, in practice i.e., in the real world of software engineering, there are numerous people working in teams for multiple projects at a time. Each project uses multiple tools in their software development. Even in the case of Static Code Analysis, multiple tools are used which are each capable of addressing several types of issues. \\ \\
This shows there is a gap in knowledge need to addressed and therefore this Thesis attempts to progress in fulfilling that gap. A paper from Miss Lisa titled Gamifying Static Analysis expressed the challenges. One of which is Responsiveness of the User Interface with Static Code Analysis tools. This narrows down our Thesis work to focus on that challenge. \\ \\
To address the Responsiveness challenge, the novel ideas are tested along with the study of different Software Engineering disciplines.

